\documentclass{ximera}
      
\title{Limits and Co-Limits of the Category Grphs}
      
\begin{document}
      
\begin{abstract}
      
In this research, we studied the basics of Category Theory and Graph Theory. In particular, we studied the basic limits of a category (product, co-product, equalizer, and co-equalizer) mainly in the category of sets (\textbf{Set}). We then studied these limits in the category of graphs (\textbf{Grphs}) (vowel omitted by convention). Finally, Sage was implemented to quickly compute these limits.      
\end{abstract}
      
\maketitle

\textbf{Visualizing Graphs}      

 To start with, let's simply visualize some graphs.  Above we have graphs $G$ and $H$.  By clicking execute, you can see what those graphs look like.

 You may add/delete edges and by extension vertices by adjusting the pairs in the lists.  For example:

\begin{verbatim}
H = Graph([(1, 2), (2, 3), (3, 4), (4, 1) ])
\end{verbatim}

will turn $H$ into a 4-cycle with edges following "sequentially".     

\begin{sageCell}
G = Graph([('a', 'b'),('b', 'c'), ('b', 'd')])
H = Graph([(1, 2), (2, 3) ])

#!!!!!!!!!!!!!!!!!!!!!!!!!!!!!!!!!!!!!!!!!!!!!!!!!!!!!!!!!!!
#PLEASE DO NOT EDIT ANYTHING AFTER THIS LINE!!
#!!!!!!!!!!!!!!!!!!!!!!!!!!!!!!!!!!!!!!!!!!!!!!!!!!!!!!!!!!!!

print("\n\n Graph G \n")
G.show()

print("\n\n Graph H \n\n")
H.show()
\end{sageCell}


\textbf{Products and CoProducts}

Next, we have the limit Product and colimit CoProduct.  Again simply modify G and H to your liking (as above) and hit run to see what the Product and Coproduct look like!


\begin{sageCell}
#Graphs G and H
G = Graph([('a', 'b'),('b', 'c'), ('d','a'), ('b', 'd')])
H = Graph([(1, 2), (2, 3) ])

#!!!!!!!!!!!!!!!!!!!!!!!!!!!!!!!!!!!!!!!!!!!!!!!!!!!!!!!!!!!!!!!!!!!!!
#PLEASE DO NOT EDIT ANYTHING AFTER THIS LINE!
#!!!!!!!!!!!!!!!!!!!!!!!!!!!!!!!!!!!!!!!!!!!!!!!!!!!!!!!!!!!!!!!!!!!!!

print("\n\n Graph G \n")
G.show()


print("\n\n Graph H \n\n")
H.show()

#Prints the Product and Coproduct
print("Grphs product of G and H")
Prod = G.strong_product(H); Prod.show()

print("Grphs co-product of G and H")
Coprod = G.disjoint_union(H); Coprod.show()

\end{sageCell}

\textbf{The Hom Sets}

You may again edit graphs $G$ and $H$.   Here, $G$ represents a domain graph and $H$ a codomain graph.  The code takes all possible functions from the vertices and edges of $G$ to the vertices and edges of $H$, and identify those that represent legitimate graph morphisms.  This is Hom$(G, H)$.  We also identify the subset of strict graph morphisms.  This is HomSt$(G,H)$.


\begin{sageCell}
#Graphs G and H, G is the Domain, H is the coDomain
G = Graph([('a', 'b'),('b', 'c'), ('b', 'd')]); 
H = Graph([(1, 2), (2, 3) ]); 

#!!!!!!!!!!!!!!!!!!!!!!!!!!!!!!!!!!!!!!!!!!!!!!!!!!!!!!!!!!!!!!!!
#PLEASE DO NOT EDIT ANYTHING AFTER THIS LINE
#!!!!!!!!!!!!!!!!!!!!!!!!!!!!!!!!!!!!!!!!!!!!!!!!!!!!!!!!!!!!!!!!

print("\n\n Graph G \n")
G.show()



print("\n\n Graph H \n\n")
H.show()


Pg = Set (G.vertices() + G.edges(labels=False)); 


Ph = Set (H.vertices() + H.edges(labels=False)); 

X = FiniteSetMaps(Pg, Ph); 

Z = Set(X)

HomS = Set([]); 

for f in Z:
    s = True

#Checking whether f is a graph morphism (vertices)

    for j in G.vertices():
        if f(j)in H.vertices():
            s=s and True
        else:
            s=s and False


#Checking whether f is a graph morphism (edges)

    for i in G.edges(labels=False):
        u=i[0];
        v=i[1];
        if f(u)==f(v) and f(u)==f(i):
            s=s and True
        elif f(i)in H.edges(labels=False):
            z=f(i);
            if (z[0]==f(u) and z[1]==f(v)) or (z[1]==f(u) and z[0]==f(v)):
                s=s and True
            else:
                s=s and False
        else: s=s and False

#Adds f to HomS if f is a morphism
    if s==True:
        HomS = HomS + Set([f])

#We now display Hom(G, H)

print("\n\n Hom(G, H) \n\n")


for j in HomS:
    print j

#Creating the Strict Homset

HomSt = Set([])

for g in HomS:
    s = True
    for v in G.edges(labels=false):
        if g(v) in H.vertices():
            s = False
    if s == True:
        HomSt = HomSt + Set([g])

#Displaying the Strict Homset

print("\n\n HomSt(G, H) \n\n")


for j in HomSt:
    print j
\end{sageCell}


\textbf{Randomized Equalizer and CoEqualizer}

Once again edit $G$ and $H$ to your liking, they are again the domain and codomain.  The code takes two random morphisms from Hom$(G, H)$ and creates the Equalizer limit and CoEqualizer colimit.


\begin{sageCell}
#Graphs G and H  G is the Domain, H is the coDomain.
G = Graph([('a', 'b'),('b', 'c'), ('b','d')])
H = Graph([(1, 2), (2, 3) ])

#!!!!!!!!!!!!!!!!!!!!!!!!!!!!!!!!!!!!!!!!!!!!!!!!!!!!!!!!!!!!!!!!!!!!!!!!!!!!!!!!
#PLEASE DO NOT EDIT ANYTHING AFTER THIS LINE!
#!!!!!!!!!!!!!!!!!!!!!!!!!!!!!!!!!!!!!!!!!!!!!!!!!!!!!!!!!!!!!!!!!!!!!!!!!!!!!!!!

print("\n\n Graph G \n")
G.show()



print("\n\n Graph H \n\n")
H.show()

#Creating the Part Sets for each graph


Pg = Set (G.vertices() + G.edges(labels=False)); 


Ph = Set (H.vertices() + H.edges(labels=False)); 

X = FiniteSetMaps(Pg, Ph); 

Z = Set(X)

HomS = Set([]); 

for f in Z:
    s = True

#Checking whether f is a graph morphism (vertices)

    for j in G.vertices():
        if f(j)in H.vertices():
            s=s and True
        else:
            s=s and False


#Checking whether f is a graph morphism (edges)

    for i in G.edges(labels=False):
        u=i[0];
        v=i[1];
        if f(u)==f(v) and f(u)==f(i):
            s=s and True
        elif f(i)in H.edges(labels=False):
            z=f(i);
            if (z[0]==f(u) and z[1]==f(v)) or (z[1]==f(u) and z[0]==f(v)):
                s=s and True
            else:
                s=s and False
        else: s=s and False

#Adds f to HomS if f is a morphism
    if s==True:
        HomS = HomS + Set([f])

#Picking Random Morphisms

Homfg = FiniteEnumeratedSet(HomS)

f = Homfg.random_element()
print("\n\n Morphism f \n\n")
f
g = Homfg.random_element()
print("\n\n Morphism g \n\n")
g





#Eq is the equalizer graph

Eq = Graph()

#Creating the equalizer


for j in G.vertices():
    if f(j)==g(j):
        Eq.add_vertex(j)


for j in G.edges(labels=False):
    u=j[0]
    v=j[1]
    if f(j)==g(j) and j[0] in Eq.vertices() and j[1] in Eq.vertices():
        Eq.add_edge(j)


print("\n\n Graph Eq \n\n")

Eq.show()

#Coeq is the co-equalizer graph

Coeq = Graph(multiedges=True, loops=True)

Gdone = Set([])
Hdone = Set([])

for p in H.vertices():
    if p not in Hdone:
        Gtemp=Set([])
        Htemp=Set([p])
        s = False
        while s == False:
            Gcheck=Gtemp
            Hcheck=Htemp
            for x in Pg:
                if f(x) in Htemp:
                    Gtemp = Gtemp + Set([x])
                if g(x) in Htemp:
                    Gtemp = Gtemp + Set([x])
            for p in Gtemp:
                Htemp = Htemp + Set([f(p), g(p)])

            if (Gtemp == Gcheck) and (Htemp == Hcheck):
                s = True
                Gdone = Gdone + Gtemp
                Hdone = Hdone + Htemp
                Coeq.add_vertex(Htemp)


for p in H.edges(labels=False):
    if p not in Hdone:
        Gtemp=Set([])
        Htemp=Set([p])
        s = False
        while s == False:
            Gcheck=Gtemp
            Hcheck=Htemp
            for x in Pg:
                if f(x) in Htemp:
                    Gtemp = Gtemp + Set([x])
                if g(x) in Htemp:
                    Gtemp = Gtemp + Set([x])
            for p in Gtemp:
                Htemp = Htemp + Set([f(p), g(p)])

            if (Gtemp == Gcheck) and (Htemp == Hcheck):
                s = True
                Gdone = Gdone + Gtemp
                Hdone = Hdone + Htemp
                e=Htemp[0]
                for v in Coeq.vertices():
                    if e[0] in v:
                        a = v
                    if e[1] in v:
                        b = v
                Coeq.add_edge(a, b, Htemp)


print("\n\n Graph Coeq \n\n")

Coeq.show()
\end{sageCell}


\textbf{Equalizer and CoEqualizer}

Finally, you may choose your own $f$ and $g$ as well as $G$ and $H$.  Use the format as follows.

Suppose:

\begin{verbatim}
G = Graph([('a', 'b'),('b', 'c'), ('b', 'd')])
H = Graph([(1, 2), (2, 3) ])
\end{verbatim}

Then:

\begin{verbatim}
f = X.from_dict({'a':1, 'b':2, 'c':3, 'd':3, ('a', 'b'):(1, 2), ('b', 'c'):(2, 3), ('b', 'd'):(2, 3)}); 
\end{verbatim}

Is a morphism from $G$ to $H$, that maps vertices $a\to1, b\to2, c,d\to3$ and edges $(a,b)\to(1,2)$ and $(b,c), (b,d)\to(2,3)$.  Another example:

\begin{verbatim}
g = X.from_dict({'a':1, 'b':2, 'c':1, 'd':3, ('a', 'b'):(1, 2), ('b', 'c'):(1, 2), ('b', 'd'):(2, 3)}); 
\end{verbatim}

Is a morphism from G to H, that maps vertices $a, c\to1, b\to2, d\to3$ and edges $(a,b), (b,c)\to$(1,2) and $(b,d)\to(2,3)$.

Again the code generates the Equalizer and CoEqualizer.  If your functions do not indicate legitimate morphisms, a message will display.


\begin{sageCell}
?b8a68cda-b183-4c89-a809-30655ad4ccca?
import sys
from sage.all import *

#Graphs G and H  G is the Domain, H is the coDomain.
G = Graph([('a', 'b'),('b', 'c'), ('b', 'd')])
H = Graph([(1, 2), (2, 3) ])

##DO NOT EDIT THESE 3 LINES
Pg = Set (G.vertices() + G.edges(labels=False));
Ph = Set (H.vertices() + H.edges(labels=False));
X = FiniteSetMaps(Pg, Ph)
##DO NOT EDIT THE ABOVE 3 LINES


f = X.from_dict({'a':1, 'b':2, 'c':3, 'd':3, ('a', 'b'):(1, 2), ('b', 'c'):(2, 3), ('b', 'd'):(2, 3)}); 


g = X.from_dict({'a':1, 'b':2, 'c':1, 'd':3, ('a', 'b'):(1, 2), ('b', 'c'):(1, 2), ('b', 'd'):(2, 3)}); 



#!!!!!!!!!!!!!!!!!!!!!!!!!!!!!!!!!!!!!!!!!!!!!!!!!!!!!!!!!!!!!!!!!!!!!!!!!!!!!!!!
#PLEASE DO NOT EDIT ANYTHING AFTER THIS LINE!
#!!!!!!!!!!!!!!!!!!!!!!!!!!!!!!!!!!!!!!!!!!!!!!!!!!!!!!!!!!!!!!!!!!!!!!!!!!!!!!!!

print("\n\n Graph G \n")
G.show()



print("\n\n Graph H \n\n")
H.show()


s = True

#Checking whether f and g are graph morphisms (vertices) It seems like this only checks if f is a graph morphism rather than g. - Dr. C

for j in G.vertices():
    if f(j)in H.vertices():
        s=s and True
    else:
        s=s and False


#Checking whether f and g are graph morphisms (edges).

for i in G.edges(labels=False):
    u=i[0];
    v=i[1];
    if f(u)==f(v) and f(u)==f(i):
        s=s and True
    elif f(i)in H.edges(labels=False):
        z=f(i)
        if (z[0]==f(u) and z[1]==f(v)) or (z[1]==f(u) and z[0]==f(v)):
            s=s and True
        else:
            s=s and False
    else: s=s and False

print("\n\n Are f and g Graph Morphism? (T/F) \n\n"); s


#Breaks the program if it is not a graph morphism

if s == False:
   quit("Not a Morphism")

#Eq is the equalizer graph

Eq = Graph()

#Creating the equalizer


for j in G.vertices():
    if f(j)==g(j):
        Eq.add_vertex(j)


for j in G.edges(labels=False):
    u=j[0]
    v=j[1]
    if f(j)==g(j) and j[0] in Eq.vertices() and j[1] in Eq.vertices():
        Eq.add_edge(j)


print("\n\n Graph Eq \n\n")
Eq.show()


#Coeq is the co-equalizer graph

Coeq = Graph(multiedges=True, loops=True)

Gdone = Set([])
Hdone = Set([])

for p in H.vertices():
    if p not in Hdone:
        Gtemp=Set([])
        Htemp=Set([p])
        s = False
        while s == False:
            Gcheck=Gtemp
            Hcheck=Htemp
            for x in Pg:
                if f(x) in Htemp:
                    Gtemp = Gtemp + Set([x])
                if g(x) in Htemp:
                    Gtemp = Gtemp + Set([x])
            for p in Gtemp:
                Htemp = Htemp + Set([f(p), g(p)])

            if (Gtemp == Gcheck) and (Htemp == Hcheck):
                s = True
                Gdone = Gdone + Gtemp
                Hdone = Hdone + Htemp
                Coeq.add_vertex(Htemp)


for p in H.edges(labels=False):
    if p not in Hdone:
        Gtemp=Set([])
        Htemp=Set([p])
        s = False
        while s == False:
            Gcheck=Gtemp
            Hcheck=Htemp
            for x in Pg:
                if f(x) in Htemp:
                    Gtemp = Gtemp + Set([x])
                if g(x) in Htemp:
                    Gtemp = Gtemp + Set([x])
            for p in Gtemp:
                Htemp = Htemp + Set([f(p), g(p)])

            if (Gtemp == Gcheck) and (Htemp == Hcheck):
                s = True
                Gdone = Gdone + Gtemp
                Hdone = Hdone + Htemp
                e=Htemp[0]
                for v in Coeq.vertices():
                    if e[0] in v:
                        a = v
                    if e[1] in v:
                        b = v
                Coeq.add_edge(a, b, Htemp)


print("\n\n Graph CoEq \n\n")
Coeq.show()
\end{sageCell}


















\end{document}
