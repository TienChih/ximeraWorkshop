% Try adding the option [handout]

\documentclass{ximera}
\title{A Sample Activity}
\author{The Ximera Team}

\begin{document}

\begin{abstract}
This is a starting point for you to create your own activity.
\end{abstract}

\maketitle

My name is Jim Fowler.  I am editing this file.  Whee.

I can edit it!  And as I edit it, it will update.

\begin{problem}
   The tolerance (0.01) means $2 \approx \answer[tolerance=0.01]{2}$
\end{problem}

 \begin{problem}
   The tolerance (0.01) means $\pi \approx \answer[tolerance=0.01]{3.141592653}$
 \end{problem}

 \begin{problem}
   The tolerance (17) means $3421 \approx \answer[tolerance=17]{3421}$
 \end{problem}


 \begin{problem}
   We can add such as in $2 + 2 = \answer{4}$.
   \begin{problem}
     Multiplication looks like $3 \times 3 = \answer{9}$.
     \begin{problem}
       Now consider $\sqrt{\answer{16}} = 4$.
       \begin{problem}
         Therefore $4 \times 4 = \answer{16}$.
       \end{problem}
     \end{problem}
   \end{problem}
   \end{problem}

\begin{problem}
  \begin{multipleChoice}
     \choice{Incorrect}
     \choice{Wrong}
     \choice[correct]{It's this one}
     \choice{Not Right}
   \end{multipleChoice}

   \begin{problem}
     There were $\answer{4}$ possible answers to that question.

     \begin{problem}
       \begin{multipleChoice}
         \choice{Not correct}
         \choice[correct]{Pick me!}
         \choice{False}
         \choice{Untrue}
       \end{multipleChoice}
     \end{problem}
   \end{problem}
\end{problem}

\end{document}
